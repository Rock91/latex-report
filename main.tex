\documentclass{article}

% Language setting
\usepackage[english]{babel}

% Set page size and margins
\usepackage[letterpaper,top=2cm,bottom=2cm,left=3cm,right=3cm,marginparwidth=1.75cm]{geometry}

% Useful packages
\usepackage{amsmath}
\usepackage{graphicx}
\usepackage[colorlinks=true, allcolors=blue]{hyperref}
\usepackage{cite}

\title{Text Mining}
\author{Ronak Sutariya}

\begin{document}
\maketitle

\begin{abstract}
Text mining is the process of extracting useful information from unstructured text data using various techniques such as natural language processing and machine learning. This report provides an overview of text mining and its applications in different fields. It also discusses two important mathematical equations used in text mining, the TF-IDF and cosine similarity equations, along with their respective applications.
\end{abstract}

\section{Introduction}
Text Mining is the process of extracting meaningful information from unstructured textual data. With the exponential growth of digital content, Text Mining has become an essential tool for information retrieval, information extraction, and knowledge discovery. Text Mining uses various techniques from Natural Language Processing (NLP), Machine Learning (ML), and Data Mining to analyze and understand textual data. In this report, we will explore the concept of Text Mining, its techniques, applications, and its importance in today's world.

Many manual and tedious tasks can be eliminated with the help of text mining. Suppose you need to understand how the customers feel about a software application you offer. Of course, you can manually go through user reviews, but if there are thousands of reviews, the process becomes tedious and time-consuming.

\section{Techniques}

Text Mining involves various techniques to process unstructured textual data. Some of the common techniques used in Text Mining are:

\subsection{Tokenization}

Tokenization is the process of breaking down a piece of text into smaller units, called tokens. These tokens can be individual words, phrases, or even entire sentences.

\subsection{Stopword Removal}

Stopword removal is a technique used in text mining to remove common words that are not considered useful for analysis. These words, known as stopwords, are typically very common words like "the", "and", "a", "an", "in", "is", and so on.

\subsection{Stemming}

Stemming is a technique used in text mining to reduce words to their base or root form, called a stem. This is done to group together words that have the same root, which can help in reducing the size of the vocabulary and improving the accuracy of text analysis.

\subsection{Lemmatization}

Lemmatization is a technique used in text mining to reduce words to their base form, called a lemma. Unlike stemming, which reduces words to a root form by removing prefixes and suffixes, lemmatization uses a dictionary or morphological analysis to determine the lemma of a word based on its part of speech and context.

\subsection{Named Entity Recognition (NER)}

Named Entity Recognition (NER) is a technique used in text mining to identify and extract named entities from unstructured text data. A named entity is a word or phrase that represents a specific object, person, place, organization, or other entity.

\subsection{Sentiment Analysis}

Sentiment analysis is a technique used in text mining to identify and extract subjective information from text data, such as opinions, emotions, and attitudes.

\subsection{Topic Modeling}

Topic modeling is a technique used in text mining to discover hidden themes or topics in large collections of text data. The goal of topic modeling is to automatically identify the underlying topics in the text data and cluster similar documents together.

\subsection{Word Embeddings}

Word embeddings are a technique used in natural language processing (NLP) and text mining to represent words as vectors of numerical values in a high-dimensional space. Word embeddings are trained on large collections of text data using machine learning algorithms to capture the semantic and syntactic relationships between words.

\section{Applications}

Text mining has various applications in different fields such as social media, healthcare, finance, and e-commerce. For example, in healthcare, text mining can be used to extract valuable information from patient records, clinical trials, and scientific literature to improve patient care and treatment. In finance, text mining can be used to analyze news articles, social media posts, and other sources of information to predict market trends and investment opportunities. Some of the common applications of Text Mining are:
\begin{itemize}
\item{Customer Relationship Management (CRM)}

Analyzing customer feedback and complaints to improve customer experience.
\item{Social Media Analysis}

Analyzing social media data to understand public opinion and sentiment.
\item{Healthcare}

Analyzing electronic health records to identify disease patterns and treatment outcomes.
\item{Fraud Detection}

Analyzing financial data to detect fraudulent transactions.
\end{itemize}

\section{Equations}
\subsection{The most common equation used in Text Mining is the Term Frequency-Inverse Document Frequency (TF-IDF) equation, which is given by:}
TF-IDF stands for term frequency-inverse document frequency. It is a statistical measure that is used to evaluate the importance of a word in a document.
Let w is a term or word,d is a document, D is a set of all documents, TF(w,d) be the frequency of term w in document d, and IDF(w, D) be determined by log (N / n).
let
\[TF-IDF(w,d,D) = TF(w,d)\times{IDF(w,D)}\]
Where N is the total number of documents and n is the number of documents containing the term w.

The TF-IDF equation calculates a weight for each word in a document based on its frequency in the document and the inverse document frequency. The higher the TF-IDF value, the more important the word is in the document.

\subsection{The Second most common equation used in Text Mining is the Term Cosine Similarity equation, which is given by:}
Cosine similarity is a measure of the similarity between two documents. It is based on the cosine of the angle between two vectors in a high-dimensional space. The cosine similarity equation is as follows:
\[cosine similarity(d1,d2) =\dfrac{(d1\times{d2})}{(\left \| d1 \right \|\times{\left \| d2 \right \|})}\]

where,
d1 and d2 are two documents.
$\mid  \mid d1 \mid\mid$ and $\mid\mid d2 \mid\mid$ are the magnitudes of the vectors d1 and d2, respectively.

The cosine similarity equation calculates a similarity score between two documents based on the angle between their vectors in a high-dimensional space. The score ranges from 0 to 1, where 1 means the documents are identical and 0 means they are completely dissimilar.


\section{Conclusion:}
Text mining is a valuable technique for analyzing and extracting insights from unstructured text data. The TF-IDF and cosine similarity equations are important mathematical concepts used in text mining. These equations help to evaluate the importance of words in documents and calculate the similarity between two documents, respectively. Text mining has numerous applications in different fields, making it a growing research area.

\begin{filecontents}{sample.bib}
@article {1,
    author  = "Berry, M. J. A, John Wiley \& Sons",
    title   = "Data Mining Techniques",
    journal = "Data Mining Techniques",
    year    = "2012",
    note    = "For Marketing, Sales, and Customer Support."
}
@article {2,
    author  = "Feldman, R., \& Sanger, J.",
    title   = "The text mining handbook",
    journal = "The text mining handbook",
    year    = "2007",
    note    = "advanced approaches in analyzing unstructured data."
}
\end{filecontents}
Dummy citation \cite{article, 1, 2}
\bibliographystyle{plain}

\end{document}